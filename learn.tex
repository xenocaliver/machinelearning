%%This is a very basic article template.
%%There is just one section and two subsections.
\documentclass[12pt]{jreport}
\usepackage{amsmath}  %AMSTeX機能を利用
\usepackage{amsfonts}
\usepackage{amssymb}  %AMSTeXの記号を利用
\usepackage{framed}
\usepackage{mathrsfs} %AMSTeXの花文字を利用
\usepackage{bm}       %イタリック体+ボールド
\usepackage{indentfirst}
\usepackage{amsbsy}   %bold italic体のためのパッケージ
\usepackage{eufrak}   %ドイツ語ひげ文字
\usepackage{dsfont}   %ds font for unit matrix
\usepackage{theorem}
\usepackage{enumerate}
\usepackage[dvipdfmx]{graphicx}
%行列記述用パッケージ
\usepackage{bigdelim}
\usepackage{multirow}
\usepackage{array,arydshln}
%枠用スタイルファイル
\usepackage{ascmac}
%改良版verbatim環境用スタイルファイル
\usepackage{moreverb}
%jreportでは参考文献が参考図書になってしまうのを修正するコマンド
\renewcommand{\bibname}{参考文献}
%for schematic diagram
\usepackage[all]{xy}
% for tensor indices
\usepackage{tensor}
% for wick contraction
%\usepackage{simplewick}
% for big integral
\usepackage{relsize}
% for floating figures extensions
\usepackage{float}
% for tabularx
\usepackage{tabularx}
% for simultaneous equations
\usepackage{empheq}
% page settings
\setlength{\textwidth}{45zw}      % テキスト幅: 
\hoffset=-25mm
\makeatother
%%%%%% for theorem setting use package amsthm.sty%%%%%
%%%%%% Please refer to http://osksn2.hep.sci.osaka-u.ac.jp/~naga/miscellaneous/tex/tex-tips5.html %%%%%
% newtheorem declarations
\theoremstyle{break}
\newtheorem{theorem}{定理}[chapter]
\newtheorem{lemma}[theorem]{補題}
\newtheorem{corollary}[theorem]{系}
\newtheorem{claim}[theorem]{主張}
\newtheorem{remark}[theorem]{注意}
\newtheorem{example}[theorem]{例}
\newtheorem{definition}[theorem]{定義}
\newtheorem{proposition}[theorem]{命題}
\newtheorem{assumption}[theorem]{仮定}
\newtheorem{condition}[theorem]{条件}
\newtheorem{exproblem}[theorem]{例題}
\newtheorem{exercise}{問題}[chapter]

\newcommand{\qed}{$\blacksquare$}
% define solution environment equivalent to proof environment
%%http://d.hatena.ne.jp/abenori/20120108
\newenvironment{solution}[1][解答]{\par
\vskip.5\Cvs
\noindent
{\bfseries #1}\par
}{\hfill ■\par
\vskip.5\Cvs
}
\newenvironment{proof}[1][証明:]{\par
\vskip.5\Cvs
\noindent
{\bfseries #1}\par
}{\hspace*{\fill}\qed\par
\vskip.5\Cvs
}
%italic Gamma
\def\itGamma{\mathit{\Gamma}}
% italic Omega
\def\itOmega{\mathit{\Omega}}
% italic Delta
\def\itDelta{\mathit{\Delta}}
% italic Lambda
\def\itLambda{\mathit{\Lambda}}
% non-zero 2dim complex space
\def\nonzerocomplexspace{\mathbb{C}^{2}\backslash\{(0,0)\}}
% regular shief
\def\calo{\mathcal{O}}
% caligraphic M
\def\calm{\mathcal{M}}
% caligraphic F
\def\calf{\mathcal{F}}
% caligrapchic G
\def\calg{\mathcal{G}}
% caligraphic H
\def\calh{\mathcal{H}}
% universal cover
\def\coveru{\mathcal{U}}
%caligraphic U
\def\calu{\mathcal{U}}
% another universal cover
\def\coverv{\mathcal{V}}
% a set consist of differential forms
\def\cala{\mathcal{A}}
% constant on Lie algebra commutator
\def\caln{\mathcal{N}}
% image symbol
\def\image{\operatorname{Img}}
% projective line
\def\pline{\mathbb{P}^{1}}
% natural number set
\def\bN{\mathbb{N}}
% complex number field
\def\bC{\mathbb{C}}
% integer number field
\def\bZ{\mathbb{Z}}
% rational number field
\def\bQ{\mathbb{Q}}
% real number field
\def\bR{\mathbb{R}}
% support
\def\support{\operatorname{supp}}
% order 
\def\ord{\operatorname{ord}}
% Homorphism
\def\Hom{\operatorname{Hom}}
% Kernel
\def\Ker{\operatorname{Ker}}
% Autmoiphism
\def\Aut{\operatorname{Aut}}
% Divisor
\def\div{\operatorname{div}}
% Endomorphism
\def\End{\operatorname{End}}
%Trace
\def\Tr{\operatorname{Tr}}
%adjoint representation
\def\ad{\operatorname{ad}}
%bold greek symbols
\def\balpha{\boldsymbol{\alpha}}
\def\bbeta{\boldsymbol{\beta}}
\def\bgamma{\boldsymbol{\gamma}}
\def\bdelta{\boldsymbol{\delta}}
\def\blambda{\boldsymbol{\lambda}}
\def\bomega{\boldsymbol{\omega}}
\def\bsigma{\boldsymbol{\sigma}}
\def\bxi{\boldsymbol{\xi}}
\def\btheta{\boldsymbol{\theta}}
\def\bomega{\boldsymbol{\omega}}
\def\dsone{\mathds{1}}
\def\cald{\mathcal{D}}
\def\Tr{\operatorname{Tr}}
\def\itLambda{\mathit{\Lambda}}
\def\itOmega{\mathit{\Omega}}
\def\itPhi{\mathit{\Phi}}
\def\hsymb#1{\mbox{\strut\rlap{\smash{\Huge$#1$}}\quad}}
%%%%%%%%%%%%%
\title{\bf{機械学習理論入門}}
\date{\hfill}
\author{\bf{橋本 顕義}}
\begin{document}
\maketitle
\tableofcontents

\chapter*{はじめに}
\addcontentsline{toc}{chapter}{はじめに}
近年,機械学習の最新技法である,deep learningなどが華々しい成果をあげて,機械学習に注目が
集まっている。本レポートでは機械学習に関する基礎理論を紹介する。本レポートは東北大学
丹内隼也氏の修士論文「確率伝搬法を用いた統計的学習理論に関する研究」の主要項目を
抜粋したものである。学習理論の初学者に好適な論文を執筆してくださった丹内氏にこの場を
借りてお礼を申し上げる。

\chapter{ボルツマンマシン}
ボルツマンマシンとは,脳の神経の基本要素であるニューロン素子が相互に結合した
集合である。統計物理学の基礎を築いた物理学者であるが,その名の由来は以下の
議論で明らかになる。

\section{ボルツマンマシンの定義}
$N$個のニューロン素子を考える。以下,ニューロン素子を単に素子と呼ぶ。
この素子の集合のエネルギーを
\begin{equation}
E_{B}(\bm{S}|\btheta,\bm{w})\triangleq\sum_{i\in\itOmega}\theta_{i}S_{i}
-\sum_{i,j\in L}w_{ij}S_{i}S_{j}\label{eq:2.1}
\end{equation}
と定義する。ここで,$\itOmega=\{1,2,\ldots,N\}$は素子の番号の集合,
$\bm{S}=\{S_{i}|i\in\itOmega\}$は素子$i$の状態を表す変数であり,
$\pm 1$の値を取るバイナリ変数とする。$L$は互いに結合している素子対のラベル
の集合である。$\btheta=\{\theta_{i}|i\in\itOmega\}$,
$\bm{w}=\{w_{ij}|i,j\in L\}$はそれぞれ閾値,結合荷重と呼ばれる
モデルパラメータであり,実際の学習においてはこれらのパラメータが調整される。
結合荷重には$w_{ii}=0,w_{ij}=w_{ji}$が仮定される。
ここまでみれば分かるが,このエネルギーは磁性の模型のイジング模型の一般形
になっていることがわかる。

ボルツマンマシンにおいて各素子$i$は時刻$t$における状態$\{S_{i}(t)\}$から,
以下の確率に従って状態を遷移させる。
\begin{equation}
p_{i}(S_{i}(t+1)=1)\triangleq\frac{1}{1+\exp[-2\beta u_{u}(t)]}
\label{eq:2.2}
\end{equation}
$\beta$はある正の定数であり,$u_{i}$は
\begin{equation}
u_{i}(t)\triangleq\sum_{j\in\partial i}w_{ij}S_{j}(t)+\theta_{i}
\label{eq:2.3}
\end{equation}
で定義されている。ここで,$\partial i$は素子$i$と結合している素子の番号である。
各時刻$t$においてランダムに素子を選び出し,式$(\ref{eq:2.2})$に従って状態を
遷移させていくとボルツマンマシンは初期状態に関わらず,次の平衡分布にたどり着く。
\begin{equation}
P_{B}(\bm{S}|\btheta,\bm{w})=\frac{1}{Z_{B}(\btheta,\bm{w})}
\exp\left[-\beta E_{B}(\bm{S}|\btheta,\bm{w})\right]
\label{eq:2.4}
\end{equation}
ここで,
\begin{equation}
Z_{B}(\btheta,\bm{w})\triangleq\sum_{\bm{S}}\exp
\left[-\beta E_{B}(\bm{S}|\btheta,\bm{w})\right]
\label{eq:2.5}
\end{equation}
は規格化定数である。また,
\begin{equation}
\sum_{\bm{S}}=\sum_{S_{1}=\pm 1}\sum_{S_{2}=\pm 1}\cdots\sum_{S_{N}=\pm 1}
\nonumber
\end{equation}
の意味であり,全確率変数の全配位に対する和となっている。分布$(\ref{eq:2.4})$は
ギブス・ボルツマン分布となっており,これがボルツマンマシンと呼ばれる理由である。

ボルツマンマシンを用いた学習はモデルパラメータを調整し,データの発生源となっている未知の分布
を再現することで達成される。ボルツマンマシンは構造的に多層パーセプトロン\cite{rosenblatt1958perceptron}
の働きを含んでいると考えられ,マルコフ確率場の形式をもち,豊富な表現力を有するため,
パターン認識などさまざまな問題への適用が期待されている。

ここで,ボルツマンマシンがマルコフ確率場に属するモデルであることを示す。マルコフ確率場に属する
確率分布
\begin{equation}
P(\bm{S})\propto\left(\prod_{i\in\itOmega}f_{i}(S_{i})\right)
\left(\prod_{i,j\in L}f_{ij}(S_{i},S_{j})\right)\label{eq:2.6}
\end{equation}
を考える。これは,
\begin{equation}
P(\bm{S})\propto\exp\left[\sum_{i\in\itOmega}\log f_{i}(S_{i})
+\sum_{i,j\in L}\log f_{ij}(S_{i},S_{j})\right]\label{eq:2.7}
\end{equation}
と変形できる。$S_{i}=\pm 1$とすると,一般性を失わずに
\begin{eqnarray}
\log
f_{i}(S_{i})&=&\frac{S_{i}}{2}\log\frac{f_{i}(1)}{f_{i}(-1)}\label{eq:2.8}\\
\log f_{ij}(S_{i},S_{j})&=&\frac{S_{i}S_{j}}{4}\log
\frac{f_{ij}(1,1)f_{ij}(-1,-1)}{f_{ij}(1,-1)f_{ij}(-1,1)}\nonumber\\
&+&\frac{S_{i}}{4}\log\frac{f_{ij}(1,1)f_{ij}(1,-1)}{f_{ij}(-1,-1)f_{ij}(-1,1)}
+\frac{S_{j}}{4}\log\frac{f_{ij}(1,1)f_{ij}(-1,1)}{f_{ij}(-1,-1)f_{ij}(1,-1)}
\nonumber\\
&+&\frac{1}{4}\log
f_{ij}(1,1)f_{ij}(1,-1)f_{ij}(-1,1)f_{ij}(-1,-1)\label{eq:2.9}
\end{eqnarray}
と表すことができる。 これらを式$(\ref{eq:2.7})$に代入し,整理すると
\begin{eqnarray}
P(\bm{S})&\propto&\exp\left[\sum_{i\in\itOmega}
\left(\frac{1}{2}\log\frac{f_{i}(1)}{f_{i}(-1)}
+\sum_{i\in\partial i}\frac{1}{4}
\frac{f_{ij}(1,1)f_{ij}(1,-1)}{f_{ij}(-1,-1)f_{ij}(-1,1)}\right)S_{i}\right.
\nonumber\\
&+&\sum_{i,j\in L}
\left.\left(\frac{1}{4}\log\frac{f_{ij}(1,1)f_{ij}(-1,-1)}{f_{ij}(1,-1)f_{ij}(-1,1)}\right)
S_{i}S_{j}\right]\label{eq:2.10}
\end{eqnarray}
となる。ここで,
\begin{eqnarray}
\beta\theta_{i}&=&\frac{1}{2}\log\frac{f_{i}(1)}{f_{i}(-1)}+
\sum_{j\in\partial i}\frac{1}{4}\log\frac{f_{ij}(1,1)f_{ij}(1,-1)}{
f_{ij}(-1,1)f_{ij}(-1,1)}\label{eq:2.11}\\
\beta w_{ij}&=&\frac{1}{4}
\log\frac{f_{ij}(1,1)f_{ij}(-1,-1)}{f_{ij}(1,-1)f_{ij}(-1,1)}
\label{eq:2.12}
\end{eqnarray}
と対応させることにより,分布$(\ref{eq:2.6})$はボルツマンマシン$(\ref{eq:2.4})$と
等価であることがわかる。ボルツマンマシンはマルコフ確率場に属するモデルである。
次節ではボルツマンマシンの学習則を導く。
\section{ボルツマンマシンの学習}
ボルツマンマシンに含まれる素子をそれぞれ,可視素子$V$と隠れ素子$H$の集合に分ける。
すなわち,$\itOmega=V\cup H$である。可視素子はそれぞれデータに対応する入出力の
素子であり,隠れ素子はボルツマンマシンの内部自由度を向上させる中間素子である。
実用的には隠れ素子がある場合が重要であるが,理論の簡便さやそれほど自由度を
必要としないと思われる課題に対してはしばしば隠れ素子なし$(H=\varnothing)$
のボルツマン素子が考えられることがある。

ボルツマンマシンの学習は$N$次元データ$\bm{d}^{\mu}=\{d_{i}^{\mu}|i\in V\}$
を$M$セット$\{\bm{d}^{\mu}|\mu=1,2,\ldots,M\}$を観測したとき,隠れ素子
$\bm{S}_{H}=\{S_{i}|i\in H\}$によって周辺化された分布${\displaystyle
\sum_{\bm{S}_{H}}P_{B}(\bm{S}|\btheta,\bm{w})}$がデータの経験分布
\begin{equation}
P_{0}(\bm{S}_{V})\triangleq\frac{1}{M}\sum_{\mu=1}^{M}
\prod_{i\in V}\delta(S_{i},d_{i}^{\mu})\label{eq:2.13}
\end{equation}
に近くなるようにパラメータ$\btheta,\bm{w}$を調整することで達成される。
ここで$\bm{S}_{V}=\{S_{i}|i\in V\}$であり,$\delta(x,y)$
はクロネッカーのデルタ関数を表している。つまり,カルバック・ライブラー情報量
(以下,KL情報量)
\begin{equation}
K(P_{0}\parallel P_{B})\triangleq\sum_{\bm{S}_{V}}
P_{0}(\bm{S}_{V})\log\frac{P_{0}(\bm{S}_{V})}
{\displaystyle\sum_{\bm{S}_{H}}P_{B}(\bm{S}|\btheta,\bm{w})}
\label{eq:2.14}
\end{equation}
を最小にするパラメータ$\btheta,\bm{w}$を求めることで達成される。

\subsection{隠れ素子なしのボルツマンマシンの学習則}
まず,より簡単な場合である隠れ素子なし$(H=\varnothing)$のボルツマンマシン
の学習則を導出し,その後で隠れ素子ありのボルツマンマシンの学習則を導出する。
隠れ素子がない場合は$\itOmega=V$であるから考えるべきKL情報量は
\begin{eqnarray}
K_{V}(P_{0}||P_{B})&=&\sum_{\bm{S}_{V}}\log\frac{P_{0}(\bm{S}_{V})}
{P_{B}(\bm{S}_{V}|\btheta,\bm{w})}\nonumber\\
&=&\sum_{\bm{S}_{V}}P_{0}(\bm{S}_{V})\log P_{0}(\bm{S}_{V})-
\sum_{\bm{S}_{V}}P_{0}(\bm{S}_{V})\log
P_{B}(\bm{S}_{V}|\btheta,\bm{w})\nonumber\\
&=&\sum_{\bm{S}_{V}}P_{0}(\bm{S}_{V})\log P_{0}(\bm{S}_{V})
+\beta\sum_{\bm{S}_{V}}P_{0}(\bm{S}_{V})E_{B}(\bm{S}|\btheta,\bm{w})
+\log Z_{B}(\btheta,\bm{w})\label{eq:2.15}
\end{eqnarray}
となる。パラメータに関してKL情報量$(\ref{eq:2.15})$の極値条件をとると
\begin{eqnarray}
\frac{\partial K_{V}(P_{0}\parallel P_{B})}{\partial\theta_{i}}
&=&\beta\sum_{\bm{S}_{V}}S_{i}P_{B}(\bm{S}_{V}|\btheta,\bm{w})
-\frac{\beta}{M}\sum_{\mu=1}^{M}d_{i}^{\mu}=0\label{eq:2.16}\\
\frac{\partial K_{V}(P_{0}\parallel P_{B})}{\partial w_{ij}}
&=&\beta\sum_{\bm{S}_{V}}S_{i}S_{j}P_{B}(\bm{S}_{V}|\btheta,\bm{w})
-\frac{\beta}{M}\sum_{\mu=1}^{M}d_{i}^{\mu}d_{j}^{\mu}=0
\label{eq:2.17}
\end{eqnarray}
となる。この極値条件から,隠れ素子なしのボルツマンマシンの学習則は最急降下法を用いて
\begin{eqnarray}
\theta_{i}(t+1)&=&\theta_{i}(t)-\beta\eta_{i}\left(
\sum_{\bm{S}_{V}}S_{i}P_{B}(\bm{S}_{V}|\btheta,\bm{w})-\frac{1}{M}
\sum_{\mu=1}^{M}d_{i}^{\mu}\right)\label{eq:2.18}\\
w_{ij}(t+1)&=&w_{ij}(t)-\beta\eta_{ij}\left(
\sum_{\bm{S}_{V}}S_{i}S_{j}P_{B}(\bm{S}_{V}|\btheta,\bm{w})
-\frac{1}{M}\sum_{\mu=1}^{M}d_{i}^{\mu}d_{j}^{\mu}\right)
\label{eq:2.19}
\end{eqnarray}
と定式化される。ここで,$(\eta_{i},\eta_{ij})$は学習率と呼ばれる小さな正の
定数であり,ステップ幅に対応している。$t$はアルゴリズムの時間ステップを表しており,
$\{\theta_{i}(t),w_{ij}(t)\}$はステップ$t$におけるパラメータの値である。
適当な初期値を与え,学習則$(\ref{eq:2.18}),(\ref{eq:2.19})$に従い
パラメータの値を更新していくことで,学習は達成される。学習則をみると$1$回の更新ごと
にボルツマン分布$P_{B}(\bm{S}_{V}|\btheta,\bm{w})$に関する期待値
を計算する必要があるが,これは一般に素子数$N$が増加するにつれて計算量が
$O(2^{N})$で増大してしまう計算困難な問題である。このため,ボルツマンマシンに
関する期待値を厳密に計算することは現実的ではなく,何らかの効率的な近似計算法
を考えなければならない。よく用いられる手法にマルコフ連鎖モンテカルロ法があり,
これを用いることで計算量を削減できるが,それでもなお長い緩和過程が必要となる。
\subsection{隠れ素子ありのボルツマンマシンの学習則}

\chapter{Contrasive Divergence法}
本章では,CD法について説明する。
\section{Contrasive Divergence法}
ボルツマンマシンを用いた学習では期待値の計算量が課題となっている。
確率的近似計算法であるマルコフ連鎖モンテカルロ法を用いるとしても長い緩和過程
が必要となる。CD 法はマルコフ連鎖モンテカルロ法をもとにした近似学習アルゴリズム
であるが,長い緩和過程を避け,高速に動作するように設計されている。しかし,その高
速な動作には条件がある。それは学習に用いるボルツマンマシンが可視素子のみで構成さ
れていること,または隠れ素子を含んでいたとしても隠れ素子同士には結合がないことで
ある。まずはボルツマンマシンが可視素子のみで構成されている場合,つまり隠れ素子な
しのボルツマンマシンにおけるCD 法について説明する。
\subsection{隠れ素子なしのボルツマンマシンの学習則}
隠れ素子なしのボルツマンマシンの学習則を導出した際に出発点となったのは
KL情報量である。
\begin{equation}
K_{V}(P_{0}\parallel P_{B})=\sum_{\bm{S}_{V}}
P_{0}(\bm{S}_{V})\log\frac{P_{0}(\bm{S}_{V})}{P_{B}(\bm{S}_{V}|\btheta,\bm{w})}
\label{eq:3.1}
\end{equation}
これに対し,CD法の更新則の導出の出発点となるのは次のContrastive Divergenceと
呼ばれる量である。
\begin{equation}
C_{V}\triangleq K_{V}(P_{0}\parallel P_{\infty})
-K_{V}(P_{n}\parallel P_{\infty})\label{eq:3.2}
\end{equation}
ここで$P_{n}$はデータの経験分布$P_{0}$からマルコフ連鎖モンテカルロ法
により$n$ステップ遷移させた分布であり,長い緩和過程を経た後の分布を
$P_{\infty}=P_{B}(\bm{S}|\btheta,\bm{w})$と表している。
ここではマルコフ連鎖モンテカルロ法の更新アルゴリズムとしてメトロポリス法
を採用する。すなわち,状態$\bm{S}$から$\bm{S}^{\prime}$に遷移させる
かどうかを判断するのに$2$つの状態の実現確率の比で決定する。すなわち,
\begin{equation}
\frac{P_{B}(\bm{S}^{\prime})}{P_{B}(\bm{S})}
=\exp\left[-2\beta S_{i}\left(\theta_{i}+\sum_{j\in\partial i}
w_{ij}S_{j}\right)\right]\label{eq:3.3}
\end{equation}
である。

では,KL情報量$K(P_{0}\parallel P_{\infty})$をパラメータについて微分すると
\begin{eqnarray}
\frac{\partial K_{V}(P_{0}\parallel P_{\infty})}
{\partial\theta_{i}}&=&\beta\langle S_{i}\rangle_{B}
-\frac{\beta}{M}\sum_{\mu=1}^{M}d_{i}^{\mu}\label{eq:3.6}\\
\frac{\partial K_{V}(P_{0}\parallel P_{\infty})}
{\partial w_{ij}}=\beta\langle S_{i}S_{j}\rangle_{B}
-\frac{\beta}{M}\sum_{\mu=1}^{M}d_{i}^{\mu}d_{j}^{\mu}
\label{eq:3.7}
\end{eqnarray}
となる。一方,KL情報量$K_{V}(P_{n}\parallel P_{\infty})$を
パラメータについて微分すると
\begin{eqnarray}
\frac{\partial K_{V}(P_{n}\parallel P_{\infty})}
{\partial\theta_{i}}&=&\beta\langle S_{i}\rangle_{B}
-\beta\langle S_{i}\rangle_{n}
+\frac{\partial P_{n}}{\partial\theta_{i}}
\frac{K_{V}(P_{n}\parallel P_{\infty})}{\partial P_{n}}\label{eq:3.8}\\
\frac{\partial K_{V}(P_{n}\parallel P_{\infty})}{\partial w_{ij}}
&=&\beta\langle S_{i}S_{j}\rangle_{B}-\beta\langle S_{i}S_{j}\rangle_{n}
+\frac{\partial P_{n}}{\partial w_{ij}}
\frac{\partial K_{V}(P_{n}\parallel P_{\infty})}{\partial P_{n}}
\label{eq:3.9}
\end{eqnarray}
となる。ここで$\langle\cdot\rangle_{n}$は確率分布$P_{n}$に関する期待値である。
式$(\ref{eq:3.6})-(\ref{eq:3.9})$の結果からContrastive Divergenceのパラメータ
に関する極値条件は
\begin{eqnarray}
\frac{\partial C_{V}}{\partial\theta_{i}}&=&\beta\langle S_{i}\rangle_{n}
-\frac{\beta}{M}\sum_{\mu=1}^{M}d_{i}^{\mu}-\frac{\partial
P_{n}}{\partial\theta_{i}}\frac{\partial K_{V}(P_{n}\parallel P_{\infty})}
{\partial P_{n}}=0\label{eq:3.10}\\
\frac{\partial C}{\partial w_{ij}}&=&\beta\langle S_{i}S_{j}\rangle_{n}
-\frac{\beta}{M}\sum_{\mu=1}^{M}d_{i}^{\mu}d_{j}^{\mu}
-\frac{\partial P_{n}}{\partial w_{ij}}\frac{\partial
K_{V}(P_{n}\parallel P_{\infty})}{\partial P_{n}}=0\label{eq:3.11}
\end{eqnarray}
となる。学習則において,$(\ref{eq:3.10}),(\ref{eq:3.11})$の第$3$項は簡便のため無視
される。この項は他の$2$項に比べて小さいことが知られている。これを省略したうえで最急降下法によって
隠れ素子のないボルツマンマシンに対するCD法の更新則は
\begin{eqnarray}
\theta_{i}(t+1)&=&\theta_{i}(t)-\beta\eta_{i}
\left(\langle S_{i}\rangle_{n}-\frac{1}{M}\sum_{\mu=1}^{M}d_{i}^{\mu}\right)
\label{eq:3.12}\\
w_{ij}(t+1)&=&w_{ij}(t)-\beta\eta_{ij}\left(\langle S_{i}S_{j}\rangle_{n}
-\frac{1}{M}\sum_{\mu=1}^{M}d_{i}^{\mu}d_{j}^{\mu}\right)
\label{eq:3.13}
\end{eqnarray}
となる。ここで注目すべきことは,これらの更新則にボルツマン分布に関する期待値が現れていない
ことである。CD法ではKL情報量の差をとることによってボルツマン分布に関する期待値の計算を回避
できている。その代わりに$\langle\cdot\rangle_{n}$が現れている。CD法は本来なら,長い
緩和時間が必要な$\langle\cdot\rangle_{B}$の計算を$\langle\cdot\rangle_{n}$
で置き換えることで計算量を削減していると解釈できる。これまでの研究で$n$は小さい数で問題なく,
$n=1$でもよい精度で学習できることが判明しており,非常に高速な近似学習アルゴリズムとなっている。
\subsection{隠れ素子ありのボルツマンマシンの学習則}
隠れ素子ありのボルツマンマシンの学習則も同様に議論できる。KL情報量を
\begin{equation}
K_{H}(\itPhi_{0}\parallel P_{B})=\sum_{\bm{S}}\itPhi_{0}(\bm{S})
\frac{\itPhi_{0}(\bm{S})}{P_{B}(\bm{S}|\btheta,\bm{w})}\label{eq:3.14}
\end{equation}
と捉え直す。ここで
\begin{eqnarray}
\itPhi_{0}(\bm{S})&\triangleq&q_{0}(\bm{S}_{H}|\bm{S}_{V},\btheta,\bm{w})P_{0}(\bm{S}_{V})
\label{eq:3.15}\\
&=&\frac{1}{M}\sum_{\mu=1}^{M}q_{0}(\bm{S}_{H})|\bm{d}^{\mu},\btheta,\bm{w})
\prod_{i\in V}\delta(S_{i},d_{i}^{\mu})\label{eq:3.16}
\end{eqnarray}
と定義する。これに対してもContrasive Divergenceを定義する。
\begin{equation}
C_{H}\triangleq K_{H}(\itPhi_{0}\parallel P_{\infty})
-K_{H}(\itPhi_{n}\parallel P_{\infty})\label{eq:3.17}
\end{equation}
$C_{H}$に関する極値条件からCD法の更新則を定式化すると以下のようになる。
\begin{eqnarray}
\theta_{i}(t+1)&=&\theta_{i}(t)-\beta\eta_{i}\left(
\sum_{\bm{S}}S_{i}\itPhi_{n}(\bm{S}|\btheta,\bm{w})
-\sum_{\bm{S}}S_{i}q_{0}(\bm{S}_{H}|\bm{S}_{V},\btheta,\bm{w})
P_{0}(\bm{S}_{V})\right)\label{eq:3.27}\\
w_{ij}(t+1)&=&w_{ij}(t)-\beta\eta_{ij}
\left(\sum_{\bm{S}}S_{i}S_{j}\itPhi_{n}(\bm{S}|\btheta,\bm{w})
-\sum_{\bm{S}}S_{i}S_{j}q_{0}(\bm{S}_{H}|\bm{S}_{V},\btheta,\bm{w})P_{0}
(\bm{S}_{V})\right)\label{eq:3.28}
\end{eqnarray}
隠れ素子がない場合と同様にボルツマンマシンに関する期待値の計算を回避できている
ことがわかる。しかしながら,括弧内第$1$項の計算量の問題はいまだ残っている。

ここでは,隠れ素子どうしが結合を持たない場合$(L_{H}=\varnothing)$である場合
に問題を限定する。そうすると,
\begin{equation}
q_{0}(\bm{S}_{H}|\bm{d}^{\mu},\btheta,\bm{w})
\propto\prod_{i\in H}\exp\left[
-\beta\left(\theta_{i}+\sum_{j\in\partial_{V}i}w_{ij}d_{j}^{\mu}\right)S_{i}
\right]\label{eq:3.35}
\end{equation}
と整理でき,隠れ素子がそれぞれ独立な確率分布になっていることがわかる。
\chapter{統計力学的近似計算法}
前章で述べたように,学習には計算量的な困難が伴うことがわかった。
本章では,このような困難をさけるため,統計力学の技法を援用した近似
計算法を解説する。
\section{統計力学的近似手法}
統計力学では自由エネルギーと呼ばれる物理量を議論の出発点とすることが
多い。自由エネルギーとは
\begin{equation}
F=-\frac{1}{\beta}\log Z=\sum_{\bm{S}}\mathcal{H}(\bm{S})
P(\bm{S})+\frac{1}{\beta}\sum_{\bm{S}}P(\bm{S})\log P(\bm{S})
\label{eq:4.1}
\end{equation}
である。ただし,
\begin{equation}
Z=\sum_{\bm{S}}\exp[-\beta\mathcal{H}(\bm{S})]\nonumber
\end{equation}
であり,状態和,分配関数と呼ばれる。自由エネルギーをパラメータについて
微分するなどすれば,物理量が得られる。

例えば,ボルツマンマシンの自由エネルギーを考える。
\begin{equation}
F_{B}=-\frac{1}{\beta}\log Z_{B}(\bm{S})\label{eq:4.2}
\end{equation}
これを$\theta_{i},w_{ij}$で微分すると
\begin{eqnarray}
\frac{\partial F_{B}}{\partial\theta_{i}}&=&-\sum_{\bm{S}}
S_{i}P_{B}(\btheta,\bm{w})=-\langle S_{i}\rangle_{B}\label{eq:4.3}\\
\frac{\partial F_{B}}{\partial w_{ij}}&=&-\sum_{\bm{S}}
S_{i}S_{j}P_{B}(\btheta,\bm{w})=-\langle S_{i}S_{j}\rangle_{B}
\label{eq:4.4}
\end{eqnarray}
となり,状態量の$1$次,$2$次モーメントの期待値が求まる。

しかし,期待値はボルツマン分布に関する期待値であり計算は困難である。さらに,その困難さは
分配関数の計算の困難さに起因している。そこで,試行確率分布$Q$を使った
\begin{equation}
\mathcal{F}[Q]=\sum_{\bm{S}}\mathcal{H}(\bm{S})Q(\bm{S})
+\frac{1}{\beta}\sum_{\bm{S}}Q(\bm{S})\log Q(\bm{S})\label{eq:4.5}
\end{equation}
という試行自由エネルギーを考える。$\mathcal{F}[Q]$を最小にする確率分布は
$Q(\bm{S})=P(\bm{S})$であり,試行自由エネルギーは
\begin{equation}
F=\min_{Q}\mathcal{F}[Q]\label{eq:4.6}
\end{equation}
という性質をもつ。統計力学ではしばしば,この試行自由エネルギーを扱いやすい形式に近似し,
その最小値を真の自由エネルギー$F$の近似値とする近似的アプローチが取られる。
この試行分布に関する自由エネルギー$\mathcal{F}[Q]$を\textbf{ベーテ自由エネルギー}
という。

ベーテ自由エネルギーを起点とする近似手法は情報科学の観点からみるとKL情報量の最小化
に対応している。ボルツマン分布$P(\bm{S})$と試行分布$Q(\bm{S})$との間の
KL情報量は
\begin{equation}
K(Q\parallel P)=\sum_{\bm{S}}Q(\bm{S})\log\frac{Q(\bm{S})}{P(\bm{S})}
=\beta(\mathcal{F}[Q]-F)\label{eq:4.7}
\end{equation}
と変形できる。真の自由エネルギー$F$は試行確率分布$Q(\bm{S})$に依存しない量
であるから$K(Q\parallel P)$の$Q(\bm{Q})$に対する最小化は試行自由エネルギー
$Q(\bm{S})$に対する最小化と直接対応していることがわかる。つまり,式$(\ref{eq:4.7})$
はKL情報量最小化を試行自由エネルギー最小化という問題に置き換えることによって,
確率分布を解析する首相に統計力学的な方法論が援用できることを表している。
\subsection{平均場近似}
平均場近似では,各素子の状態を取りうる確率分布を
\begin{equation}
Q_{i}(S_{i})\triangleq\sum_{\bm{S}\backslash S_{i}}Q(\bm{S})\nonumber
\end{equation}
と仮定する。そして,ボルツマンマシン全体の確率分布は
\begin{equation}
Q(\bm{S})=\prod_{i\in\itOmega}Q_{i}(S_{i})\label{eq:4.8}
\end{equation}
と仮定する。これは各素子は独立に周囲からの影響を受けていると仮定することにほかならない。
これを試行自由エネルギー$(\ref{eq:4.5})$に代入すると
\begin{eqnarray}
\mathcal{F}[Q]&=&\sum_{\bm{S}}\left(-\sum_{i\in\itOmega}\theta_{i}S_{i}
-\sum_{i,j\in L}w_{ij}S_{i}S_{j}\right)\prod_{i\in\itOmega}
Q_{i}(S_{i})+\frac{1}{\beta}\sum_{\bm{S}}
\left(\prod_{i\in\itOmega}Q_{i}(S_{i})\right)
\log\left(\prod_{i\in\itOmega}Q_{i}(S_{i})\right)\nonumber\\
&=&-\sum_{i\in\itOmega}\theta_{i}\sum_{S_{i}}S_{i}Q_{i}(S_{i})-
\sum_{i,j\in L}w_{ij}\sum_{S_{i}}S_{i}Q_{i}(S_{i})
\sum_{S_{j}}S_{j}Q_{j}(S_{j})
+\frac{1}{\beta}\sum_{i\in\itOmega}\sum_{S_{i}}Q_{i}(S_{i})\log(S_{i})
\label{eq:4.9}
\end{eqnarray}
となる。ここで,各素子の確率変数$S_{i}$の平均値を
\begin{equation}
m_{i}\triangleq\sum_{\bm{S}}S_{i}Q(\bm{S})=\sum_{S_{i}}S_{i}Q_{i}(S_{i})
\label{eq:4.10}
\end{equation}
と定義する。$S_{i}=\pm 1$であることから一般性を失わずに
\begin{equation}
Q_{i}(S_{i})=\frac{1+m_{i}S_{i}}{2}\label{eq:4.11}
\end{equation}
と表せる。これをベーテ自由エネルギー$(\ref{eq:4.10})$に代入すると
\begin{equation}
\mathcal{F}[Q]=\mathcal{F}(\bm{m})=-\sum_{i\in\itOmega}
\theta_{i}m_{i}-\sum_{i,j\in L}w_{ij}m_{i}m_{j}
+\frac{1}{\beta}\sum_{i\in\itOmega}\sum_{S_{i}}\frac{1+S_{i}m_{i}}{2}
\log\frac{1+S_{i}m_{i}}{2}\label{eq:4.12}
\end{equation}
となる。この極値条件は
\begin{eqnarray}
\frac{\partial\mathcal{F}(\bm{m})}{\partial m_{i}}&=&
-\theta_{i}-\sum_{j\in\partial i}w_{ij}m_{ij}+\frac{1}{\beta}
\sum_{S_{i}}\frac{S_{i}}{2}\log\frac{1+S_{i}m_{i}}{2}\nonumber\\
&=&-\theta_{i}-\sum_{j\in\partial i}w_{ij}m_{j}+\frac{1}{\beta}
\tanh^{-1}m_{i}=0\label{eq:4.13}
\end{eqnarray}
である。これを整理すると,
\begin{equation}
m_{i}=\tanh\left[\beta\left(\theta_{i}+\sum_{i\in\partial
i}w_{ij}m_{j}\right)\right]\label{eq:4.14}
\end{equation}
となる。これは平均場方程式と呼ばれ,$m_{i}$をニュートン法などで数値的に
解くことができる。平均場近似はその簡便さにより利用が容易であり,汎用性が高い
近似法である。しかし,数値的な近似精度は高くない。
\subsection{確率伝搬法}
平均場近似より近似精度の高い計算手法にベーテ近似がある。ベーテ近似は
物理学の近似手法であるが,情報科学の分野で独立に考案された確率伝搬法
と等価であることが知られている\cite{kabashima2004statistical}
\cite{kabashima1999statistical}。ここでは,確率伝搬法を導出する。

まず,試行分布を次の形式に制限する。
\begin{equation}
Q(\bm{S})=\left(\prod_{i\in\itOmega}Q_{i}(S_{i})\right)
\left(\prod_{i,j\in L}\frac{Q_{ij}(S_{i},S_{j})}
{Q_{i}(S_{i})Q_{j}(S_{j})}\right)\label{eq:4.15}
\end{equation}
ただし,
\begin{eqnarray}
Q_{i}(S_{i})&\triangleq&\sum_{\bm{S}\backslash S_{i}}Q(\bm{S})\label{eq:4.16}\\
Q_{ij}(S_{i},S_{j})&\triangleq&
\sum_{\bm{S}\backslash\{S_{i},S_{j}\}}Q(\bm{S})\label{eq:4.17}
\end{eqnarray}
と定義する。この試行分布は隣接する素子間の相関を考慮した形となっている。
式$(\ref{eq:4.5})$からベーテ自由エネルギーは
\begin{eqnarray}
\mathcal{F}&=&\sum_{\bm{S}}\left(-\sum_{i\in\itOmega}\theta_{i}S_{i}
-\sum_{i,j\in L}w_{ij}S_{i}S_{j}\right)Q(\bm{S})\nonumber\\
&+&\frac{1}{\beta}\sum_{\bm{S}}Q(\bm{S})\sum_{i\in\itOmega}\log Q_{i}(S_{i})
+\frac{1}{\beta}\sum_{\bm{S}}Q(\bm{S})\sum_{i,j\in L}\log Q_{ij}(S_{i},S_{j})
\nonumber\\
&-&\frac{1}{\beta}\sum_{\bm{S}}Q(\bm{S})\sum_{i,j\in L}\left(
\log Q_{i}(S_{i})-\log Q_{j}(S_{j})\right)\nonumber\\
&=&-\sum_{i\in\itOmega}\theta_{i}\sum_{\bm{S}}S_{i}Q(\bm{S})
\sum_{i,j\in L}w_{ij}\sum_{\bm{S}}S_{i}S_{j}Q(\bm{S})\nonumber\\
&+&\frac{1}{\beta}\sum_{i\in\itOmega}(1-z_{i})\sum_{S_{i}}Q_{i}(S_{i})
+\frac{1}{\beta}\sum_{i,j\in L}\sum_{S_{i},S_{j}}Q_{ij}(S_{i},S_{j})
\log Q_{ij}(S_{i},S_{j})\label{eq:4.18}
\end{eqnarray}
となる。ここで,$z_{i}$は素子$i$と結合する素子の数であり,$z_{i}=|\partial i|$である。
試行分布に関する$S_{i}$の期待値を
\begin{eqnarray}
m_{i}&\triangleq&\sum_{\bm{S}}S_{i}Q(\bm{S})=\sum_{S_{i}}S_{i}Q_{i}(S_{i})\label{eq:4.19}\\
m_{ij}&\triangleq&\sum_{\bm{S}}S_{i}S_{j}Q(\bm{S})=\sum_{S_{i},S_{j}}S_{i}S_{j}Q_{ij}(S_{i},S_{j})\label{eq:4.20}
\end{eqnarray}
と定義すると,$S_{i}=\pm 1$であることから一般性を失わずに
\begin{eqnarray}
Q_{i}(S_{i})&=&\frac{1+S_{i}m_{i}}{2}\label{eq:4.21}\\
Q_{ij}(S_{i},S_{j})&\triangleq&\frac{1+S_{i}m_{i}+S_{j}m_{j}+S_{i}S_{j}m_{ij}}{4}
\label{eq:4.22}
\end{eqnarray}
とできる。このことからベーテ自由エネルギーは
\begin{eqnarray}
\mathcal{F}&=&-\sum_{i\in\itOmega}\theta_{i}m_{i}-\sum_{i,j\in L}w_{ij}m_{ij}
+\frac{1}{\beta}\sum_{i\in\itOmega}(1-z_{i})\sum_{S_{i}}\frac{1+S_{i}m_{i}}{2}
\log\frac{1+S_{i}m_{i}}{2}\nonumber\\
&+&\frac{1}{\beta}\sum_{i,j\in
L}\sum_{S_{i},S_{j}}\frac{1+S_{i}m_{i}+S_{j}m_{j}+S_{i}S_{j}m_{ij}}{4}
\log\frac{1+S_{i}m_{i}+S_{j}m_{j}+S_{i}S_{j}m_{ij}}{4}\label{eq:4.23}
\end{eqnarray}
となり,$\{m_{i},m_{ij}\}$の関数として整理される。

ベーテ自由エネルギーの$\{m_{i},m_{ij}\}$に関する最小値を求めよう。
$(\ref{eq:4.23})$を$m_{i}$で微分すると
\begin{eqnarray}
\frac{\partial\mathcal{F}}{\partial m}&=&-\theta_{i}+
\frac{1-z_{i}}{\beta}\sum_{S_{i}}\frac{S_{i}}{2}\log\frac{1+S_{i}m_{i}}{2}
+\frac{1}{\beta}\sum_{j\in\partial i}\sum_{S_{i},S_{j}}\frac{S_{i}}{4}
\log\frac{1+S_{i}m_{i}+S_{j}m_{j}+S_{i}S_{j}m_{ij}}{4}\nonumber\\
&=&-\theta_{i}+\frac{1-z_{i}}{\beta}\tanh^{-1}m_{i}+\frac{1}{2\beta}
\sum_{j\in\partial
i}\sum_{S_{j}}\tanh^{-1}\frac{m_{i}+S_{j}m_{ij}}{1+S_{j}m_{j}}\nonumber\\
&=&-\theta_{i}+\frac{1-z_{i}}{\beta}\tanh^{-1}m_{i}
+\frac{1}{2\beta}\sum_{i\in\partial i}\left(
\tanh^{-1}\frac{m_{i}+m_{ij}}{1+m_{j}}+\tanh^{-1}\frac{m_{i}-m_{ij}}{1-m_{ij}}
\right)\label{eq:4.24}
\end{eqnarray}
となる。したがって,極値条件は
\begin{equation}
\theta_{i}=\frac{1-z_{i}}{\beta}\tanh^{-1}m_{i}+\frac{1}{2\beta}
\sum_{j\in\partial i}\left(
\tanh^{-1}\frac{m_{i}+m_{ij}}{1+m_{j}}+\tanh^{-1}\frac{m_{i}-m_{ij}}{1-m_{ij}}
\right)\label{eq:4.25}
\end{equation}
である。これを変形すると,$z_{i}\neq 1$のとき
\begin{equation}
m_{i}=\tanh\left[\frac{\beta}{1-z_{i}}\left\{\theta_{i}-
\frac{1}{2\beta}\sum_{j\in\partial i}
\left(
\tanh^{-1}\frac{m_{i}+m_{ij}}{1+m_{j}}+\tanh^{-1}\frac{m_{i}-m_{ij}}{1-m_{ij}}
\right)\right\}\right]
\end{equation}
が得られる。これは$m_{i}$を決める方程式となる。また,式$(\ref{eq:4.23})$を$m_{ij}$で微分
すると
\begin{equation}
\frac{\partial\mathcal{F}}{\partial m_{ij}}=
-w_{ij}+\frac{1}{\beta}\sum_{S_{i},S_{j}}\frac{S_{i}S_{j}}{4}
\log\frac{1+S_{i}m_{i}+S_{j}m_{j}+S_{i}S_{j}m_{ij}}{4}\nonumber
\end{equation}
を得る。ここで,公式
\begin{equation}
\frac{1}{2}\log\frac{1+x}{1-x}=\tanh^{-1}x\label{eq:atanh}
\end{equation}
を使う。$x=m_{i}+S_{j}m_{ij}$と考えれば,
\begin{eqnarray}
\frac{\partial\mathcal{F}}{\partial m_{ij}}&=&-w_{ij}+\frac{1}{2\beta}
\sum_{S_{j}}S_{j}\tanh^{-1}\frac{m_{i}+S_{j}m_{ij}}{1+S_{j}m_{j}}\nonumber\\
&=&-w_{ij}+\frac{1}{2\beta}\left(
\tanh^{-1}\frac{m_{i}+m_{ij}}{1+m_{j}}-\tanh^{-1}\frac{m_{i}-m_{ij}}{1-m_{j}}
\right)\label{eq:4.27}
\end{eqnarray}
を得る。極値条件は
\begin{equation}
w_{ij}=\frac{1}{2\beta}\left(]
\tanh^{-1}\frac{m_{i}+m_{ij}}{1+m_{j}}-\tanh^{-1}\frac{m_{i}-m_{ij}}{1-m_{j}}
\right)\label{eq:4.28}
\end{equation}
である。これをさらに$m_{ij}$を求める式に変形する。
\begin{eqnarray}
A&=&\frac{m_{i}+m_{ij}}{1+m_{j}}\nonumber\\
B&=&\frac{m_{i}-m_{ij}}{1-m_{j}}\nonumber
\end{eqnarray}
とおくと
\begin{eqnarray}
2\beta w_{ij}&=&\tanh^{-1}A-\tanh^{-1}B\nonumber\\
&=&\frac{1}{2}\log\frac{1+A}{1-A}-\frac{1}{2}\log\frac{1+B}{1-B}\nonumber
\end{eqnarray}
となる。よって
\begin{eqnarray}
4\beta w_{ij}&=&\log\frac{1+A}{1-A}-\log\frac{1+B}{1-B}\nonumber\\
&=&\log\frac{1+A}{1+B}\frac{1-B}{1-A}\nonumber
\end{eqnarray}
となる。したがって
\begin{equation}
e^{4\beta w_{ij}}=\frac{1+A}{1+B}\frac{1-B}{1-A}\nonumber
\end{equation}
である。
\begin{equation}
(1+B)(1-A)e^{4\beta w_{ij}}=(1+A)(1-B)\nonumber
\end{equation}
\begin{equation}
(1-A+B-AB)e^{4\beta w_{ij}}=1+A-B-AB\nonumber
\end{equation}
\begin{equation}
AB(e^{4\beta w_{ij}}-1)+(e^{4\beta w_{ij}}+1)(A-B)-e^{4\beta
w_{ij}}+1=0\nonumber
\end{equation}
\begin{equation}
AB+\cot(2\beta w_{ij})(A-B)-1=0\nonumber
\end{equation}
である。ここで
\begin{eqnarray}
AB&=&\frac{m_{i}^{2}-m_{ij}^{2}}{1-m_{j}^{2}}\nonumber\\
A-B&=&\frac{(1-m_{j})(m_{i}+m_{ij})-(1+m_{j})(m_{i}-m_{ij})}{1-m_{j^{2}}}\nonumber\\
&=&\frac{m_{i}-m_{i}m_{j}+m_{ij}-m_{j}m_{ij}-m_{i}-m_{i}m_{j}+m_{ij}+m_{j}m_{ij}}
{1-m_{j}^{2}}\nonumber\\
&=&\frac{2m_{ij}-2m_{i}m_{j}}{1-m_{j}^{2}}\nonumber
\end{eqnarray}
であるから
\begin{equation}
m_{i}^{2}-m_{ij}^{2}+\cot(2\beta
w_{ij})(2m_{ij}-2m_{i}m_{j})-1+m_{j}^{2}=0\nonumber
\end{equation}
\begin{equation}
m_{ij}^{2}-2\cot(2\beta w_{ij})m_{ij}+2m_{i}m_{j}+1-m_{i}^2-m_{j}^{2}=0\nonumber
\end{equation}
である。この$m_{ij}$に関する二次方程式を解くと
\begin{equation}
m_{ij}=\cot(2\beta w_{ij})\left\{1\pm
\sqrt{1-(1-m_{i}^{2}-m_{j}^{2})\tanh^{2}(2\beta
w_{ij})-2m_{i}m_{j}\tanh^{2}(2\beta w_{ij})}\right\}\label{eq:4.29}
\end{equation}
を得る。$|m_{ij}|\leqq 1$という条件から
\begin{equation}
m_{ij}=\cot(2\beta w_{ij})\left\{1-
\sqrt{1-(1-m_{i}^{2}-m_{j}^{2})\tanh^{2}(2\beta
w_{ij})-2m_{i}m_{j}\tanh^{2}(2\beta w_{ij})}\right\}\label{eq:4.30}
\end{equation}
が得られる。これが$m_{ij}$を決める式となる。

これらから$m_{i},m_{ij}$を数値的に求めることが原理的にはできるはずであるが,さらに簡単な
方式を求める。まず,
\begin{equation}
\mathcal{M}_{j\rightarrow i}\triangleq-\frac{1}{2}
\left(\tanh^{-1}\frac{m_{i}+m_{ij}}{1+m_{j}}+\tanh^{-1}\frac{m_{i}-m_{ij}}{1-m_{j}}
\right)+\tanh^{-1}m_{i}\label{eq:4.32}
\end{equation}
という量を定義する。$(\ref{eq:4.25})$より
\begin{equation}
\tanh^{-1}m_{i}=\beta\theta_{i}+\sum_{j\in\partial i}\mathcal{M}_{j\rightarrow
i}
\label{eq:4.31}
\end{equation}
であるから,$\mathcal{M}_{j\rightarrow i}$がわかれば,$m_{i}$がわかることになる。またしても
公式$(\ref{eq:atanh})$を使うと
\begin{equation}
\mathcal{M}_{i\rightarrow j}=-\frac{1}{4}\log
\frac{1+m_{j}+m_{i}+m_{ij}}{1-m_{j}+m_{i}-m_{ij}}
-\frac{1}{4}\log\frac{1+m_{j}-m_{i}-m_{ij}}{1-m_{j}-m_{i}+m_{ij}}
-\frac{1}{2}\log\frac{1+m_{j}}{1-m_{j}}\nonumber
\end{equation}
ここで,
\begin{eqnarray}
1+m_{j}&=&a\nonumber\\
1-m_{j}&=&b\nonumber\\
m_{i}+m_{ij}&=&c\nonumber\\
m_{i}-m_{ij}&=&d\nonumber
\end{eqnarray}
とおくと,
\begin{eqnarray}
\mathcal{M}_{i\rightarrow
j}&=&-\frac{1}{4}\log\frac{a+c}{b+d}-\log\frac{a-c}{b-d}+\frac{1}{2}\log\frac{a}{b}
\nonumber\\
&=&\frac{1}{4}\log\frac{b+d}{a+c}+\frac{1}{4}\log\frac{b-d}{a-c}+\frac{1}{2}\log\frac{a}{b}
\nonumber\\
&=&\frac{1}{2}\log\left(\sqrt{\frac{b+d}{a+c}\frac{b-d}{a-c}}\frac{a}{b}\right)\nonumber\\
&=&\frac{1}{2}\log\frac{\displaystyle\frac{2a}{\sqrt{a^{2}-c^{2}}}}
{\displaystyle\frac{2b}{\sqrt{b^{2}-d^{2}}}}\nonumber
\end{eqnarray}
ここで分子は
\begin{equation}
\frac{\sqrt{2a}}{\sqrt{a^{2}-c^{2}}}=\frac{a+c+a-c}{\sqrt{a^{2}-c^{2}}}
=\sqrt{\frac{a+c}{a-c}}+\sqrt{\frac{a-c}{a+c}}\nonumber
\end{equation}
となる。分母も同様で,
\begin{eqnarray}
\mathcal{M}_{i\rightarrow j}&=&\frac{1}{2}\log
\frac{\displaystyle\sqrt{\frac{a+c}{a-c}}+\sqrt{\frac{a-c}{a+c}}}
{\displaystyle\sqrt{\frac{b+d}{b-d}}+\sqrt{\frac{b-d}{b+d}}}\nonumber\\
&=&\frac{1}{2}\log\frac{\displaystyle\cosh\left(\frac{1}{2}\log\frac{a+c}{a-c}\right)}
{\displaystyle\cosh\left(\frac{1}{2}\log\frac{b+d}{b-d}\right)}\nonumber
\end{eqnarray}
となる。ここでまたしても,公式$(\ref{eq:atanh})$を使うと
\begin{equation}
\mathcal{M}_{i\rightarrow j}=\frac{1}{2}\log\frac{\displaystyle
\cosh\left(\tanh^{-1}\frac{c}{a}\right)}
{\displaystyle\cosh\left(\tanh^{-1}\frac{d}{b}\right)}
=\frac{1}{2}\log
\frac{\displaystyle\cosh\left(\tanh^{-1}\frac{m_{i}+m_{ij}}{1+m_{j}}\right)}
{\displaystyle\cosh\left(\tanh^{-1}\frac{m_{i}-m_{ij}}{1-m_{j}}\right)}
\label{eq:4.33}
\end{equation}
が得られる。また,式$(\ref{eq:4.28})$は
\begin{equation}
\tanh^{-1}\frac{m_{i}+m_{ij}}{1+m_{j}}-\tanh^{-1}\frac{m_{i}-m_{ij}}{1-m_{j}}
=2\beta w_{ij}\label{eq:4.34}
\end{equation}
と変形でき,式$(\ref{eq:4.31}),(\ref{eq:4.32})$より
\begin{eqnarray}
\tanh^{-1}\frac{m_{i}+m_{ij}}{1+m_{j}}-\tanh^{-1}\frac{m_{i}-m_{ij}}{1-m_{j}}
&=&2\tanh^{-1}m_{i}-2\mathcal{M}_{i\rightarrow j}\nonumber\\
&=&2\beta\theta_{i}+2\sum_{k\partial i\backslash j}\mathcal{M}_{k\rightarrow i}
\label{eq:4.35}
\end{eqnarray}
となる。式$(\ref{eq:4.34}),(\ref{eq:4.35})$を連立させると
\begin{eqnarray}
\tanh^{-1}\frac{m_{i}+m_{ij}}{1+m_{j}}&=&\beta\theta_{i}+\beta w_{ij}
+\sum_{k\in\partial i\backslash j}\mathcal{M}_{k\rightarrow i}\label{eq:4.36}\\
\tanh^{-1}\frac{m_{i}-m_{ij}}{1-m_{j}}&=&\beta\theta_{i}-\beta w_{ij}
+\sum_{k\in\partial i\backslash j}\mathcal{M}_{k\rightarrow i}\label{eq:4.37}
\end{eqnarray}
を得る。これを式$(\ref{eq:4.33})$に代入すると
\begin{eqnarray}
\mathcal{M}_{i\rightarrow j}&=&\frac{1}{2}\log
\frac{\displaystyle\cosh\left(\beta\theta_{i}+\beta w_{ij}
+\sum_{k\in\partial i\backslash j}\mathcal{M}_{k\rightarrow i}\right)}
{\displaystyle\cosh\left(\beta\theta_{i}-\beta w_{ij}
+\sum_{k\in\partial i\backslash j}\mathcal{M}_{k\rightarrow i}\right)}\nonumber
\end{eqnarray}
となる。ここで,双曲線関数の加法定理
\begin{equation}
\cosh(\alpha\pm\beta)=\cosh(\alpha)\cosh(\beta)\pm\sinh(\alpha)\sinh(\beta)
\nonumber
\end{equation}
を使う。そうすると
\begin{equation}
\mathcal{M}_{i\rightarrow j}=\frac{1}{2}\log\frac{
\displaystyle 1+\tanh(\beta w_{ij})\tanh\left(\beta\theta_{i}+
\sum_{k\in\partial i\backslash j}\mathcal{M}_{k\rightarrow i}\right)}
{\displaystyle 1-\tanh(\beta w_{ij})\tanh\left(\beta\theta_{i}+
\sum_{k\in\partial i\backslash j}\mathcal{M}_{k\rightarrow i}\right)}
\label{eq:4.39}
\end{equation}
となる。さらに公式$(\ref{eq:atanh})$を使うと
\begin{equation}
\mathcal{M}_{i\rightarrow j}=\tanh^{-1}\left[
\tanh(\beta w_{ij})\tanh\left(\beta\theta_{i}+
\sum_{k\in\partial i\backslash j}\mathcal{M}_{k\rightarrow i}\right)
\right]\label{eq:4.40}
\end{equation}
と変形できる。この式がメッセージ$\{\mathcal{M}_{i\rightarrow j}$を決める方程式となり,
これを数値的に解くことによりメッセージ$\{\mathcal{M}_{i\rightarrow j}\}$の値を決める
ことができる。これが決まれば,式$(\ref{eq:4.31})$より
\begin{equation}
m_{i}=\tanh\left(\beta\theta_{i}+\sum_{j\in\partial
i}\mathcal{M}_{j\rightarrow i}\right)\label{eq:4.41}
\end{equation}
という関係があるため,$m_{i}$が決まる。$m_{i}$が決まれば$m_{ij}$も決まる。
このようにして,ベーテ自由エネルギーの極小値を探す問題の解を探すことができる。

以上のように,統計力学の近似手法を用いることで,計算量を減らす学習則を導くことができた。
%参考文献
\bibliographystyle{junsrt} % 日本語標準スタイル
\bibliography{code}
%\appendix
\end{document}
